%ELTE Biológia Címlap Latex

% magyar nyelv
\documentclass[a4paper,12pt]{article}
\usepackage[magyar]{babel}
\usepackage[utf8]{inputenc}

% margók stb. beállítása
\usepackage{geometry}
\geometry{
    left=25mm,
    right=25mm,
    top=25mm,
    bottom=25mm,
}

% paraméterek definiálása
% cím
\newcommand{\thesistitle}[1]{

	\vspace{48pt}
	\textbf{\noindent \huge {#1}}
	\vspace{24pt}

}

% szakirány
\newcommand{\spec}[1]{

		\Large Diplomamunka \\
		\large biológus mesterszak \\
		\large {#1} szakirány
		\vspace{48pt}
}

% név
\newcommand{\name}[1]{
	\large{készítette:} \\
	\Large\textbf{{#1}}
	\vspace{84pt}
	

}

% témavezető adatai
\newcommand{\consulent}[3]{
	\large{#1}, {#2} \\
	\large{{#3}}
}

% címer
\usepackage{graphicx}
\graphicspath{ {img/} }
\usepackage{float}
\newcommand{\footer}[1]{

	\textsc{Eötvös Loránd Tudományegyetem}\\
	\textsc{Természettudományi Kar}\\
	\textsc{Biológiai Intézet} \\
	
	\begin{figure}[H]
		\centering
		\includegraphics[scale=0.35]{img/elte_cimer_vector.pdf}
	\end{figure}
	
	Budapest,#1


}

\begin{document}

	% ne legyen számozva az oldal
	\thispagestyle{empty}
	
	\begin{center}
			\thesistitle{Bakteriális patogén és ember közötti molekuláris hálózatok vizsgálata}
			
			\spec{Molekuláris Genetika, Sejt- és Fejlődésbiológia}
			
			\name{Horváth Balázs}
			
			\large Témavezetők:
			\vspace{0.21cm}
			
			\consulent{KADLECSIK TAMÁS}{PhD hallgató}{ELTE Biológiai Intézet, Genetikai Tanszék}
			
			\vspace{16pt}
			
			\consulent{DR. KORCSMÁROS TAMÁS}{csoportvezető}{ELTE, Genetikai Tanszék} \\
			\large TGAC, The Genome Analysis Centre, Norwich, Egyesült Királyság
			
			\vspace{38pt}
			
			\footer{2016}
			
	\end{center}

	
\end{document}